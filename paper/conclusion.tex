In conclusion, we have developed an early version of a scalable framework for computing approximate solutions to fractional PDEs and presented some results for some simple geometries. This is the first step towards more interesting solvers based on the method of eigenfunction expansion. 

In addition to computing the eigenbasis in a scalable, efficient manner, we have laid the groundwork for a compression scheme that will both reduce the amount of data required for such a method as well as reducing the amount of work for applying the eigenbasis. While zfp didn't give quite the results we would prefer, the next step would be to explore geometry-aware compression schemes along the lines of FMM. One of the benefits of such a scheme would be that the geometry doesn't change on an eigenvector by eigenvector basis, any of the geometry-related data structures would only need to be computed once for the mesh. In addition to this, a data structure for compressing and decompressing eigenvectors as needed would need to be developed and tested.

While the eigenbasis computation is in a scalable and efficient state, without GPU acceleration and a task manager, we are leaving computing resources on the table so to speak. Further development on this framework will need to include these features in order to achieve peak efficiency on heterogeneous computing nodes.

Looking ahead, we hope to explore more complicated fractional PDEs such as space-fractional Diffusion-Reaction or Cahn-Hiliard. This framework will act as the foundation for these explorations and should allow for very large-scale experiments to be done to verify existing and future theoretical work in this field.