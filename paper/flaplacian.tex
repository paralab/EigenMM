{\color{blue} \noindent <Define fractional poisson problem> }

{\color{blue} \noindent <Define spectral approach to solving fractional poisson problem> }

{\color{blue} \noindent <Describe non-local nature of fractional properties and what aspects can and can not be captured by the spectral approach> }